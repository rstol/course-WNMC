\documentclass[12pt,a4paper]{article}
\usepackage[utf8]{inputenc}
\usepackage{amsmath}
\usepackage{amsfonts}
\usepackage{amssymb}
%\usepackage{tikz}
\usepackage{graphicx}
%\usetikzlibrary{positioning}
\usepackage{tabularx}

%\usepackage{algorithm}
%\usepackage[noend]{algpseudocode}
%\def\BState{\State\hskip-\ALG@thistlm}

\usepackage{hyperref}
\usepackage{color}
\hypersetup{
	colorlinks,
	filecolor=black,
	linkcolor=black,
	urlcolor=black
}
\usepackage{listings}
\usepackage{xcolor}
\definecolor{codegreen}{rgb}{0,0.6,0}
\definecolor{codegray}{rgb}{0.5,0.5,0.5}
\definecolor{codepurple}{rgb}{0.58,0,0.82}
\definecolor{backcolour}{rgb}{0.95,0.95,0.92}

\usepackage{verbatim}

\lstdefinestyle{codestyle}{
    backgroundcolor=\color{backcolour},   
    commentstyle=\color{codegreen},
    keywordstyle=\color{magenta},
    numberstyle=\tiny\color{codegray},
    stringstyle=\color{codepurple},
    basicstyle=\ttfamily\footnotesize,
    breakatwhitespace=false,         
    breaklines=true,                 
    captionpos=b,                    
    keepspaces=true,                 
    numbers=left,                    
    numbersep=5pt,                  
    showspaces=false,                
    showstringspaces=false,
    showtabs=false,                  
    tabsize=2
}
\lstset{style=codestyle}

\setlength{\parindent}{0cm}
\UseRawInputEncoding
\newcommand{\listedsection}[1]{\section*{#1}\addcontentsline{toc}{section}{#1}}
\newcommand{\listedsubsection}[1]{\subsection*{#1}\addcontentsline{toc}{subsection}{#1}}

\author{David Bohner 18-951-822, Romeo Stoll 19-917-749}
\title{WNMC: Exercise 05 - VLC}
\date{Fall 2021}


\begin{document}
\maketitle
\tableofcontents
\newpage

\listedsection{Step 2}
\listedsubsection{Answers}
\begin{itemize}
\item[1.] \textbf{What is the bandwidth of the optical spectrum?}
\item[] The frequency range of visible light lies at around 430 - 750 THz. [1] This constitutes a band 320 THz in size, which by Prof. Mangold's bandwidth estimation guidelines corresponds to 320 Tb/s.
\item[2.] \textbf{Is the visible spectrum regulated?}
\item[] No.
\item[3.] \textbf{What is the difference between infrared and visible light?}
\item[] Infrared light is light at a frequency lower than the lowest frequency visible to the human eye. (This also implies a lower energy of the wave.)
\item[4.] \textbf{Can infrared light penetrate water? Can visible light?}
\item[] Infrared light \textit{can} penetrate water for a limited range, however water \textit{does} have IR-absorbing properties, making it inefficient/ineffective at long-range underwater communication. Higher-power visible light, such as blue or violet light, is less-absorbed and as such a better frequency to communicate underwater with.
\item[5.] \textbf{How do submarines communicate through the ocean, when they operate below water surface?}
\item[] Sound waves, microwaves, blue LED/lasers (i.e. high-energy visible light). (Lecture 06: VLC)
\item[6.] \textbf{How can an LED be used as a receiver?}
\item[] LEDs lose charge more quickly while in contact with photons. This allows for charge depletion rate comparisons to detect if another LED was lit up (e.g. transmitting a 1-bit, depending on implementation) or not.
\item[7.] \textbf{What are the benefits of using an LED instead of a photodiode as a receiver in consumer electronics?}
\item[] Price/practicality. Installing both an LED and a photodiode for two-way communication would be both more expensive (added product) as well as more challenging to implement. (Design: only need to place 1 LED instead of both components. Hardware: Photodiode may require a more complicated microcontroller. Price: Beyond additional design costs, the hardware of the photodiode itself obviously adds to manufacturing cost.)
\item[] Additionally, many consumer electronics already contain some kind of LED, allowing for communication implementation by just changing microcontroller programming rather than re-designing hardware.
\item[]
\item[{[1]}] https://www.britannica.com/science/color/The-visible-spectrum
\end{itemize}

\listedsection{Step 4}
\listedsubsection{Screenshot}
\includegraphics[width=\linewidth]{chatApp communication.png} 
\listedsection{Step 5}
\listedsubsection{Throughput, Delay Plot}
\includegraphics[width=\linewidth]{Figure_1.png}
\listedsubsection{Interpretation}
We begin with some clarifications: We decided to include the 10ms artificial sleep delay between \texttt{m[D]} reception and the following send in the packet delay. Less controversial is that we also considered this 'artifical' delay in our throughput analysis. We also did not include packets that were received at the destination, but for which the \texttt{ACK} was not received at the source, into the total data volume transmitted for our throughput analysis, though we did consider their (timeout) delay in the average package delay.\\
The measurements were conducted in a home-office environment with one central source of lighting in the ceiling directly overhead, hopefully giving the collected data relevancy for 'normal' home use with ambient light as a source of interference.\\\\

From our data, we (generally) observe a continuous growth in throughput \textit{and} delay as package size grew, as well as a continuous decrease in throughput and increase in delay as the separation distance was increased.\\

Throughput being proportional at some degree with packet size can be explained by the reduced percentile portion that communication overheads comprise, though starting at a distance of 35cm this effect began to diminish as the high bit failure rate begins to demand numerous re-transmissions of larger packages rather than few re-transmissions of smaller ones. This is very clearly obvious at 40cm, as after the initial increase in throughput by increasing package size from 1 to 10 bit, further increases to size cause the throughput to be reduced again (in our case, due to the very few (45\%) successfully transmitted packages, the rest running out of re-transmissions).\\

Packet delay is trivially proportional with packet size due to the increased data transmitted per packet, but also due to the higher expected number of re-transmissions needed, as explained above. It is also for the latter reason that delay increases with separation distance.
\newpage
\listedsection{Appendix}
\listedsubsection{Step 4: Chat App Source Code}
\lstinputlisting[language=Python]{chatApp.py}
\newpage

\listedsubsection{Step 5: Data Logs}
\verbatiminput{distance data.txt}
\newpage
\listedsubsection{Step 5: Source Code}
\lstinputlisting[language=Python]{distanceMeasurement.py}
\end{document}